%%%%%%%%%%%%%%%%%%%%%%%%%%%%%%%%%%%
%This is the LaTeX COMMUNICATION template for RSC journals
%Copyright The Royal Society of Chemistry 2016
%%%%%%%%%%%%%%%%%%%%%%%%%%%%%%%%%%%

\documentclass[twoside,twocolumn,9pt]{article}
\usepackage{extsizes}
\usepackage[super,sort&compress,comma]{natbib} 
\usepackage[version=3]{mhchem}
\usepackage[left=1.5cm, right=1.5cm, top=1.785cm, bottom=2.0cm]{geometry}
\usepackage{balance}
\usepackage{mathptmx}
\usepackage{sectsty}
\usepackage{graphicx} 
\usepackage{lastpage}
\usepackage[format=plain,justification=justified,singlelinecheck=false,font={stretch=1.125,small,sf},labelfont=bf,labelsep=space]{caption}
\usepackage{float}
\usepackage{fancyhdr}
\usepackage{fnpos}
\usepackage[english]{babel}
\addto{\captionsenglish}{%
  \renewcommand{\refname}{Notes and references}
}
\usepackage{array}
\usepackage{droidsans}
\usepackage{charter}
\usepackage[T1]{fontenc}
\usepackage[usenames,dvipsnames]{xcolor}
\usepackage{setspace}
\usepackage[compact]{titlesec}
\usepackage[hidelinks]{hyperref}
%%%Please don't disable any packages in the preamble, as this may cause the template to display incorrectly.%%%


% xr to cross reference files
\usepackage{xr}
\externaldocument{SI}

% https://stackoverflow.com/questions/1856189/fullpage-picture-in-two-column-layout
\usepackage{balance}

% Custom commands ---------------------------
\newcommand{\kcal}{kcal mol$^{-1}$}
\graphicspath{ {figures/} }


% ------------------------------------------------

%\usepackage{epstopdf}%This line makes .eps figures into .pdf - please comment out if not required.

\definecolor{cream}{RGB}{222,217,201}

\begin{document}

\pagestyle{fancy}
\thispagestyle{plain}
\fancypagestyle{plain}{
%%%HEADER%%%
\renewcommand{\headrulewidth}{0pt}
}
%%%END OF HEADER%%%

%%%PAGE SETUP - Please do not change any commands within this section%%%
\makeFNbottom
\makeatletter
\renewcommand\LARGE{\@setfontsize\LARGE{15pt}{17}}
\renewcommand\Large{\@setfontsize\Large{12pt}{14}}
\renewcommand\large{\@setfontsize\large{10pt}{12}}
\renewcommand\footnotesize{\@setfontsize\footnotesize{7pt}{10}}
\renewcommand\scriptsize{\@setfontsize\scriptsize{7pt}{7}}
\makeatother

\renewcommand{\thefootnote}{\fnsymbol{footnote}}
\renewcommand\footnoterule{\vspace*{1pt}% 
\color{cream}\hrule width 3.5in height 0.4pt \color{black} \vspace*{5pt}} 
\setcounter{secnumdepth}{5}

\makeatletter 
\renewcommand\@biblabel[1]{#1}            
\renewcommand\@makefntext[1]% 
{\noindent\makebox[0pt][r]{\@thefnmark\,}#1}
\makeatother 
\renewcommand{\figurename}{\small{Fig.}~}
\sectionfont{\sffamily\Large}
\subsectionfont{\normalsize}
\subsubsectionfont{\bf}
\setstretch{1.125} %In particular, please do not alter this line.
\setlength{\skip\footins}{0.8cm}
\setlength{\footnotesep}{0.25cm}
\setlength{\jot}{10pt}
\titlespacing*{\section}{0pt}{4pt}{4pt}
\titlespacing*{\subsection}{0pt}{15pt}{1pt}
%%%END OF PAGE SETUP%%%

%%%FOOTER%%%
\fancyfoot{}
\fancyfoot[LO,RE]{\vspace{-7.1pt}\includegraphics[height=9pt]{head_foot/LF}}
\fancyfoot[CO]{\vspace{-7.1pt}\hspace{13.2cm}\includegraphics{head_foot/RF}}
\fancyfoot[CE]{\vspace{-7.2pt}\hspace{-14.2cm}\includegraphics{head_foot/RF}}
\fancyfoot[RO]{\footnotesize{\sffamily{1--\pageref{LastPage} ~\textbar  \hspace{2pt}\thepage}}}
\fancyfoot[LE]{\footnotesize{\sffamily{\thepage~\textbar\hspace{3.45cm} 1--\pageref{LastPage}}}}
\fancyhead{}
\renewcommand{\headrulewidth}{0pt} 
\renewcommand{\footrulewidth}{0pt}
\setlength{\arrayrulewidth}{1pt}
\setlength{\columnsep}{6.5mm}
\setlength\bibsep{1pt}
%%%END OF FOOTER%%%

%%%FIGURE SETUP - please do not change any commands within this section%%%
\makeatletter 
\newlength{\figrulesep} 
\setlength{\figrulesep}{0.5\textfloatsep} 

\newcommand{\topfigrule}{\vspace*{-1pt}% 
\noindent{\color{cream}\rule[-\figrulesep]{\columnwidth}{1.5pt}} }

\newcommand{\botfigrule}{\vspace*{-2pt}% 
\noindent{\color{cream}\rule[\figrulesep]{\columnwidth}{1.5pt}} }

\newcommand{\dblfigrule}{\vspace*{-1pt}% 
\noindent{\color{cream}\rule[-\figrulesep]{\textwidth}{1.5pt}} }

\makeatother
%%%END OF FIGURE SETUP%%%

%%%TITLE AND AUTHORS%%%
\twocolumn[
  \begin{@twocolumnfalse}
{\includegraphics[height=30pt]{head_foot/journal_name}\hfill\raisebox{0pt}[0pt][0pt]{\includegraphics[height=55pt]{head_foot/RSC_LOGO_CMYK}}\\[1ex]
\includegraphics[width=18.5cm]{head_foot/header_bar}}\par
\vspace{1em}
\sffamily
\begin{tabular}{m{4.5cm} p{13.5cm} }

\includegraphics{head_foot/DOI} & \noindent\LARGE{\textbf{Reaction Dynamics of Diels-Alder Reactions from Machine Learned Potentials}} \\%Article title goes here instead of the text "
"
 & \vspace{0.3cm} \\

 & \noindent\large{Tom A. Young,\textit{$^{a}$} Tristan Johnson-Wood,\textit{$^{a}$}, Hanwen Zhang\textit{$^{a}$} and Fernada Duarte$^\ast$\textit{$^{a}$}} \\%Author names go here instead of "Full name", etc.

\includegraphics{head_foot/dates} & \\

\end{tabular}

 \end{@twocolumnfalse} \vspace{0.6cm}

  ]
%%%END OF TITLE AND AUTHORS%%%

%%%FONT SETUP - please do not change any commands within this section
\renewcommand*\rmdefault{bch}\normalfont\upshape
\rmfamily
\section*{}
\vspace{-1cm}


%%%FOOTNOTES%%%

\footnotetext{\textit{$^{a}$~Chemical Research Laboratory, South Parks Road, Oxford, OX1 1NQ.}}

%Please use \dag to cite the ESI in the main text of the article.
%If you article does not have ESI please remove the the \dag symbol from the title and the footnotetext below.
\footnotetext{\dag~Electronic Supplementary Information (ESI) available: [details of any supplementary information available should be included here]. See DOI: 00.0000/00000000.}
%additional addresses can be cited as above using the lower-case letters, c, d, e... If all authors are from the same address, no letter is required

% \footnotetext{\ddag~Additional footnotes to the title and authors can be included \textit{e.g.}\ `Present address:' or `These authors contributed equally to this work' as above using the symbols: \ddag, \textsection, and \P. Please place the appropriate symbol next to the author's name and include a \texttt{\textbackslash footnotetext} entry in the the correct place in the list.}

%%%END OF FOOTNOTES%%%

%%%ABSTRACT%%%%

\sffamily{\textbf{The abstract should be a single paragraph which summarises the content of the article.}}\\%The abstrast goes here instead of the text "The abstract should be..."

%%%END OF ABSTRACT%%%%

\rmfamily %Please do not remove this line.

%%%MAIN TEXT%%%%



Simulating chemical reactions is essential to developing fundamental understanding and predicting experimental outcomes.\cite{Orr-Ewing2017} Machine learned potentials (MLPs) offer an enticing approach to chemical simulation, enabling the efficient mapping between nuclear configurations and energies ($\boldsymbol{R} \mapsto E$). Moreover, they offer flexibility and systematic improvability, not possible with classical molecular mechanics (MM).\cite{Behler2016} Propagating quantum dynamics using these forces should afford experimental rate and equilibrium constants in the limit of correct forces and converged sampling. However, despite the development of Gaussian Approximation Potentials (GAPs)\cite{Bartk2010, Deringer2021} and high dimensional neural network potentials (NNPs)\cite{Behler2007} more than 10 years ago, they are still not yet routinely used to simulate chemical reactivity.\cite{Ko2020} Most likely, this is due to the computational and time investment required to train potentials for new systems. 

Training an MLP consists of: (1) developing a training set; (2) hyperparameter optimisation and (3) performing the regression, repeating the process until the desired accuracy is obtained. Automated approaches to training set construction have been developed,\cite{Smith2018, Young2021gap, Miksch2021} but can be limited to small systems or generate huge datasets. These limitations coupled with the time required to perform hyperparameter optimisation (if the MLP is insufficiently accurate) inhibits quickly accessing bespoke MLPs. Furthermore, the required $\gg 10^3$ reference evaluations precludes using accurate wavefunction-based quantum methods to evaluate energy and forces without considerable investment.10 Exceptions are rare and limited to systems with $<10$ atoms.\cite{Young2021gap, Dral2020}

For potentials suitable to simulate chemical reactivity, automated approaches are essential. The energy scale over which the potential must be accurate is larger, necessitating exponentially more training data and thus bespoke MLPs. Furthermore, the complex electronic structure around transition states makes density functional theory (DFT) a poorer reference method,\cite{Zhao2005} meaning coupled-cluster (CC) is often the target surface for quantitative comparison to experiment, which in-turn demands data-efficient strategies.

Here, we show that new MLP regression methods\cite{Batzner2021, Kovacs2021} can be used to generate accurate potentials for modestly sized reactions ($\sim50$ atoms) in an automated fashion and demonstrate the associated insights that can be obtained. 

With a view to extend our initial GAP training methodology\cite{Young2021gap} into more complex systems and environments, we considered Diels-Alder (DA) reactions because of the available theoretical and experimental data,\cite{Black2012, Lording2020} and their prominence in chemical and biochemical contexts.\cite{Sato2021, MartCentelles2018, Briou2021} Initial efforts proved promising, with qualitatively reasonable reaction dynamics from [4+2] cycloaddition TSs for reactions between ethene + butadiene20 ({\bfseries{R1}}) and methyl-vinyl ketone + cyclopentadiene ({\bfseries{R2}}). Evaluating the quality of these potentials, however, revealed that they were not within the few $k_BT$ accuracy limit required for rate estimation or dynamic studies (see e.g., Fig. {\ref{fig::SX30}}a). A similarly complex but less exothermic reaction ({\ce{H3C}· + \ce{C3H8} $\rightarrow$ \ce{CH4} + ·\ce{CH(CH3)2}}) could be trained using the same strategy and hyperparameters (Figure S1b), suggesting that achieving 1 \kcal~accuracy within a 60 \kcal~energy window required for {\bfseries{R1}} is challenging for a GAP. Hyperparameter optimisation afforded an improvement, but at moderate computational cost ($\sim500$ configurations required for {\bfseries{R1}}). Specifically, increasing the `strength’ of the fit by reducing the noise added to energies and forces, increasing the quality of the radial basis, and doubling the number of atomic environments considered in the training all improved the GAP (SI §S3). Systematic investigation of the effect of system size on the required number of reference evaluations suggests an approximate exponential scaling for a desired accuracy on the total energy (SI §\ref{section::SI_system_size}). Adopting new regression methods within the same training strategy (Fig. \ref{fig::X1}) shows that GAPs – even with hyperparameter tuning – are outperformed by both linear atomic cluster expansion (ACE\cite{Drautz2019}) and equivariant graph neural networks (NequIP\cite{Batzner2021}). 

\begin{figure*}[t]
	\centering
	\includegraphics[height=13cm]{figX1}
	%\vspace{0.1cm}
	%\hrule
	%\vspace{0.1cm}
	\caption{MLP methods trained on the [4+2] Diels-Alder cycloaddition between ethene and butadiene in the gas phase. GAP* used optimised hyperparameters (see SI). Training set developed by active learning based on MLP-dynamics with ET = 2.3 \kcal~(0.1 eV), stable potential defined by the ability to propogate 10 trajectories without finding a configuration $|E_\text{MLP} – E_0| > E_T$. (a) Signed errors on total energies over three trajectories from reactants to products (AIMD, 300K, PBE0/def2-SVP). (b) Comparison of the relative performance between MLP methods on total time, data efficiency (number of training configuration selected, $n_\text{train}$) and total accuracy. (c) Parity plot between MP2/def2-TZVP total energies and ACE predictions on MP2 AIMD trajectories from the TS. ACE trained on DFT selected configurations; energies and forces re-evaluated at MP2. (d) Comparison of the predicted (red) and true PBE0/def2-SVP (black) relaxed 2D surface surrounding the TS. Contour plot represents the ‘closeness’ of the training data to a point in the surface. .}
	\label{fig::X1}
\end{figure*}



While rather different in philosophy, both frameworks provide MLPs that are similarly accurate for {\bfseries{R1}} (Fig. \ref{fig::X1}a, Fig. \ref{fig::SX23}). Here, accuracy is based on deviations between true and predicted energies over independent DFT-MD trajectories propagated from the transition state (TS) to the reactant and product states. Previously, we have shown that a prospective validation strategy in the configuration space accessible to that MLP is essential to characterising `good’ MLPs.\cite{Young2021gap} However, here these potentials are `stable’ by construction, within their own configuration space over the course of the reaction.

This arises from the active learning strategy (Fig. \ref{fig::X1}, top) defining a stable potential where 10, 1 ps, trajectories can be propagated without encountering a configuration that is predicted to be $>2$ \kcal~(0.1 eV) above or below the true energy. Of course, checking this criterion every MD step is too computationally intensive thus stability is not guaranteed but empirically the criterion is sufficient to define stability within that period.

The data requirement to train a quality MLP for {\bfseries{R1}} is reduced upon GAP hyperparameter optimisation – even though the potential is more accurate – but is surpassed by the efficient ACE and NequIP potentials, both of which require only $\sim100$ training configurations ($n_\text{train}$, Figure X1b). The total training time is maximal for the NequIP potential but is only $\sim1/2$ day (10 cores + 1 GPU) meaning it is suitable for rapidly developing bespoke MLPs. Note that the discrepancy between the MLPs in training time reduces with the system size, with the concurrent fraction of time spent on reference energy and force evaluations (for equally data efficient MLPs). The GAPs and ACE potential required just $5\pm2$ h of total training time, on 10 CPU cores. The following sections will focus on ACE potentials for their slight advantage in computational cost.

As found for GAPs, re-evaluating energies and forces from AL configurations with a new reference method (aka. `uplifting’) reduces the computational cost associated with training WF-quality MLPs. For example, uplifting PBE0/DZ configurations to MP2/TZ affords an ACE potential within chemical accuracy to MP2-MD configurations (Fig. \ref{fig::X1}c).

Comparing the two-dimensional potential energy surface over the two forming C–C bonds where all other degrees of freedom are free to relax, reveals that the ACE potential is smooth (as are the GAP and NequIP potentials, Fig. \ref{fig::SX25}) and accurate even in the extrapolation regime (Fig. \ref{fig::X1}d). Even where the closest configuration in the training data is 0.3 \AA away in the forming bonds, the error is only a couple of \kcal~when the AL is initiated at the TS at each iteration ($r_1=r_2=2.30$ \AA).

Tangent to our goal of developing accurate MLPs for DA reactions, we found that GAP regularisation could be harnessed to reduce the computational cost of developing CCSD(T)-quality potentials (SI §S4). For simple molecules, MP2 forces are accurate to within ~0.05 eV \AA${}^{-1}$ of their CCSD(T) counterparts (Fig. \ref{fig::SX8}), thus within the `expected error’ (e.g. 0.1 eV \AA${}^{-1}$) of the GAP. This affords a dramatic speedup compared to training on numerical CCSD(T) forces. Further accelerations are possible by CUR\cite{Mahoney2009} selecting around half of the configurations from the dataset based on average SOAP kernel matrices. These effects can combine to afford a 100-fold reduction in the number of required CC calculations.

Extending the reaction complexity, the ambimodal reaction between tropone and cyclohepatriene ({\bfseries{R3}}) is capable of forming three distinct products from a single TS (Figure X2, top).\cite{Jamieson2021} Training an ACE potential from the TS generated $\sim450$ configurations using standard AL with sampling in the reactant and product regions of 2 of the 3 products. However, despite propagating ACE-MD at 500 K the Cope product (10) was not present in the training data, making it unsuitable for running dynamics to elucidate the product ratio. Only when the TS that leads directly to 10 is included is the training data sufficiently complete. This highlights the importance of considering relevant points close in energy (4.4 \kcal, ref. \cite{Jamieson2021}) when training MLPs, that may not be obtained using automated sampling methods. 

Employing this MLP to propagate dynamics enables unique observations compared to the most common DFT-MD approach. Specifically, because each 1 ps trajectory takes only a couple of minutes to calculate, the product distribution can be converged with respect to the number of trajectories (Figure X2a). Using only 100 trajectories affords an error ($2\sigma$, 95\% confidence) of 10\% for product 9, which may or may not be sufficient for experimental comparison.

The inference efficiency also enables quantum dynamics to be propagated, which otherwise be too computationally demanding to perform. Interestingly, we find that initiating ring-polymer molecular dynamics (RPMD)\cite{Habershon2013} without


%\begin{figure*}
% \centering
% \includegraphics[height=3cm]{example2}
% \caption{An image from the \textit{Physical Chemistry Chemical Physics} cover gallery, set as a double-column figure.}
% \label{fgr:example2col}
%\end{figure*}


%  \ce{H2SO4}. 

\clearpage
The conclusions section should come at the end of article. For the reference section, the style file \texttt{rsc.bst} can be used to generate the correct reference style.\footnote[4]{Footnotes should appear here. These might include comments relevant to but not central to the matter under discussion, limited experimental and spectral data, and crystallographic data.}

\section*{Author Contributions}
X

\section*{Conflicts of interest}
There are no conflicts to declare.



%%%END OF MAIN TEXT%%%

%  For footnotes in the main text of the article please number the footnotes to avoid duplicate symbols. e.g.  \footnote[num]{your text} the corresponding author \ast counts as footnote 1, ESI as footnote 2, e.g. if there is no ESI, please start at [num]=[2], if ESI is cited in the title please start at [num]=[3] etc. Please also cite the ESI within the main body of the text using \dag.

% The \balance command can be used to balance the columns on the final page if desired. It should be placed anywhere within the first column of the last page.

% \balance

% If notes are included in your references you can change the title from 'References' to 'Notes and references' using the following command:
% \renewcommand\refname{Notes and references}

%%%REFERENCES%%%
\scriptsize{
\bibliography{refs} %You need to replace "rsc" on this line with the name of your .bib file
\bibliographystyle{rsc} } %the RSC's .bst file

\end{document}
